%
% THE BEER-WARE LICENSE (Rev. 42):
% Ronny Bergmann <bergmann@math.uni-luebeck.de> wrote this file. As long as you
% retain this notice you can do whatever you want with this stuff. If we meet
% some day, and you think this stuff is worth it, you can buy me a beer or
% coffee in return.
%
% This file is just to get started - You need the corresponding Logo
%
\documentclass[german,10pt,xcolor=colortbl,compress
%,draft
]{beamer}
\usepackage{xunicode}
\usepackage[T1]{fontenc}
\usepackage{calc}
\usepackage[ngerman]{babel} % Neue Rechtschreibung
\usepackage{amsmath,amsthm,amssymb,euscript} % AMS-LaTeX  
\usepackage{enumerate,graphicx}

% Load Them
\usetheme[slogan=false, navigation=true,reducedframetotal, myriad=false]{UzL}
%
\setbeamertemplate{navigation symbols}{}
\title{Bachelorprojekt }
\subtitle{Optimierung biologisch-realistischer Neuronenmodelle}
\date[]{\today\\[1ex]Abschlusspräsentation}
\author[]{S. Bereziak \and M. Dannehl \and C. Girth \and C. Kalelioglu \and J. Wolff}
% Clear Logo 1 to make the head smaller
\institute[Universität zu Lübeck]{Institut für Robotik und kognitive Systeme\\Universität zu Lübeck}
%\clearlogo{1}
\setlogo{1}{.25\paperwidth}{logos/unilogo300dpi.png}%

\begin{document}
	\maketitle
	\begin{frame}
		\tableofcontents
	\end{frame}	
	\section{Einführung}
	\begin{frame}{Motivation}
	Wie wir uns motivieren.
	\end{frame}
	\begin{frame}{Zielsetzung}
	ABC
	\end{frame}
	
	\section{Zielsetzung}
	\begin{frame}{Ein Ziel}
	content...
	\end{frame}
	\section{Implementierung}
	\subsection{Architektur}
	
	\begin{frame}{Überblick}
	content...
	\end{frame}
	\begin{frame}{Abhängigkeiten}
	content...
	\end{frame}
	\begin{frame}{Kern}
	content...
	\end{frame}

	\begin{frame}{CLI}
	content...
	\end{frame}

	\begin{frame}{GUI}
	content...
	\end{frame}
	
	\subsection{Demo}
\end{document}
