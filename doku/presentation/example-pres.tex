%
% THE BEER-WARE LICENSE (Rev. 42):
% Ronny Bergmann <bergmann@math.uni-luebeck.de> wrote this file. As long as you
% retain this notice you can do whatever you want with this stuff. If we meet
% some day, and you think this stuff is worth it, you can buy me a beer or
% coffee in return.
%
% This file is just to get started - You need the corresponding Logo
%
\documentclass[ngerman,10pt,xcolor=colortbl,compress
%,draft
]{beamer}
\usepackage{calc}
\usepackage[ngerman]{babel} % Neue Rechtschreibung
\usepackage{amsmath,amsthm,amssymb,euscript} % AMS-LaTeX  
\usepackage{enumerate,graphicx}
\usepackage{fontspec}

% Load Them
\usetheme[slogan=false, navigation=true,reducedframetotal, myriad=false]{UzL}
%
\setbeamertemplate{navigation symbols}{}
\title{Bachelorprojekt }
\subtitle{Optimierung biologisch-realistischer Neuronenmodelle}
\date[]{14. Juli 2015\\[1ex]Abschlusspräsentation}
\author[Bereziak \and Dannehl \and Girth \and Kalelioglu \and Wolff]{S. Bereziak \and M. Dannehl \and C. Girth \and C. Kalelioglu \and J. Wolff}
% Clear Logo 1 to make the head smaller
\institute[UzL]{Institut für Robotik und kognitive Systeme\\Universität zu Lübeck}
%\clearlogo{1}
\setlogo{1}{.25\paperwidth}{logos/unilogo300dpi.png}%

\begin{document}
	\maketitle
	\begin{frame}{Gliederung}
		\tableofcontents
	\end{frame}	
	\section{Überblick}
	\begin{frame}{Überblick}{Motivation}
	\begin{itemize}
	\item Parameter von neuronalen Netzen müssen stets optimiert werden
	\item Optimierungen von Hand sind sehr aufwendig bzw. unmöglich
	\end{itemize}
	\mbox{}\\
	$\rightarrow$ Automatisierte Optimierung mittels Methoden der künstlichen Intelligenz
	\end{frame}
	\begin{frame}{Überblick}{Zielsetzung}
	Framework, bestehend aus
	\begin{itemize}
		\item Strukturen zur Kommunikation mit neuronalem Netz
		\item Flexibel implementierte Optimierungsalgorithmen
		\item Grafische Oberfläche
	\end{itemize}
	\mbox{}\\
	Weitere Projektvorgaben
	\begin{itemize}
		\item Implementierung in Python
		\item Versionskontrolle über github
	\end{itemize}
	\end{frame}
	
	\section{Implementierung}
	
	\begin{frame}{Implementierung}{Struktur}
	Hübsches Bild von Can
	\end{frame}
	\begin{frame}{Implementierung}{Kern}
	Der Kern besteht aus folgenden Komponenten
	\begin{itemize}
		\item main -- Verarbeitet Befehle von GUI oder CLI
		\item net -- Führt über SSH Netz und Analysescript aus
		\item algorithms -- Stellt Optimierungsalgorithmen bereit 
	\end{itemize}
	\mbox{}\\
	Kommunikation mit CLI bzw. GUI erfolgt über Message-Queues
	\end{frame}
	
	\begin{frame}{Implementierung}{Kern -- Implementierte Algorithmen}
	In der aktuellen Version implementierte Algorithmen
	\begin{itemize}
		\item simple\_genetic -- einfacher genetischer Algorithmus
		\item random\_search -- randomisierte Suche
	\end{itemize}
	\end{frame}
	
	\begin{frame}{Implementierung}{CLI}
	Bild $\rightarrow$ Live-Demo?
	\end{frame}

	\begin{frame}{Implementierung}{GUI}
	\begin{itemize}
		\item mainframe -- Laden der Algorithmen und Starten/Speichern der Optimierung
		\item addframe -- Auswahl der Algorithmen und Setzen der Parameter
		\item sshframe -- Eingabe der SSH-Schlüsseldaten
	\end{itemize}
	\end{frame}
	
	\begin{frame}{Implementierung}{Abhängigkeiten}
		Zur Ausführung benötigte Pakete
		\begin{itemize}
			\item python3
			\item sshpass
			\item GTK
			\item GDK
		\end{itemize}
	\end{frame}
	
	\section{Resultat}
	\begin{frame}{Resultat}{Live-Demo}
	Bild $\rightarrow$ Live-Demo
	\end{frame}

	
\end{document}
