% Poster-Vorlage angelehnt an das Corporate Design der Universität zu Lübeck
%
% geschrieben von Ronny Bergmann, 4. August 2013
%
% THE BEER-WARE LICENSE (Rev. 42):
% Ronny Bergmann <bergmann@math.uni-luebeck.de> wrote this file. As long as you
% retain this notice you can do whatever you want with this stuff. If we meet
% some day, and you think this stuff is worth it, you can buy me a beer or
% coffee in return.
%
\documentclass[german,12pt,xcolor=colortbl%
%,draft
]{beamer}
\usepackage{xunicode} 
\usepackage[orientation=portrait,size=a0,scale=1.4]{beamerposter}
\usepackage[T1]{fontenc}
\usepackage{calc}
\usepackage[ngerman]{babel} % Deutsch
\usepackage{amsmath,amsthm,amssymb,euscript} % AMS-LaTeX  
\usepackage{enumitem}
\usepackage[numbers,sort]{natbib}
\usepackage{subfigure,caption,tikz,booktabs}

% UzL-Theme laden (liegt im gleichen Verzeichnis oder unter LaTeX-Beamer)
\usetheme[footline=true, slogan=false, myriad=true, navigation=false,ITMtheme]{UzL}
% deaactivate
\setbeamertemplate{navigation symbols}{}
% Umdefinieren der Kopf- und Fußzeile
\defbeamertemplate*{headline}{UzLNew} 
{ 
\leavevmode
\vskip-4\baselineskip
\hspace{.028\paperwidth}
		\strut
		\par
		\vskip\baselineskip
		\hspace{.02\paperwidth}
}
\defbeamertemplate*{footline}{UzLNew}{%
		\centering
		\vspace{-1cm}
		\par\strut
		\begin{beamercolorbox}[wd=.3\textwidth,ht=2.75ex,left]{}%
			Hier könnte ein Projekttext stehen
		\end{beamercolorbox}
		\begin{beamercolorbox}[wd=.6\textwidth,ht=2.75ex,left]{}%
			\( ^\ast \)Institut für Mathematik\\[.5\baselineskip]
 			Universität zu Lübeck\\[.5\baselineskip]
			Ratzeburger Allee 160\\[.5\baselineskip]
 			23562 Lübeck
		\end{beamercolorbox}
		\vskip1.5\baselineskip%.012\paperheight
}
\setbeamersize{text margin left=5cm,text margin right=5cm}
\setbeamercolor{title2}{bg=uzlmain,fg=white}
\setbeamercolor{title3}{bg=uzlmain!75,fg=white}
\setlength{\parskip}{.5\baselineskip}
\setlength{\parindent}{0pt}
\newcommand{\mySec}[1]{\vspace{1\baselineskip}\par{\Large\color{uzlmain}#1}\par\vspace{.5\baselineskip}}
\begin{document}
\begin{frame}[t]%
%
% Titel
	\vspace{-1.1cm}
	\strut\par
	\includegraphics[width=273mm]{logos/institutslogo} %91*3
%	\\[.5\baselineskip]
	\vspace{.9\baselineskip}
	\begin{beamercolorbox}[wd=\textwidth,sep=\baselineskip,leftskip=-.25em]{title2}
		\parbox[t]{.95\textwidth}{\Huge Titel\\Zweite Zeile}
			\vspace{\baselineskip}
			\par
			Ronny Bergmann\( ^{\ast} \)\\{\small bergmann@math.uni-luebeck.de}\vspace{-.5\baselineskip}
		\end{beamercolorbox}
%
%
% Inhalt
 	\begin{columns}[t,totalwidth=\textwidth]
%		Erste Spalte
 		\begin{column}{.49\textwidth}
			\vspace{-.5\baselineskip}
 			\mySec{Einleitung}
			Ein erster Absatz mit Testtext und Zitat~\citep{TestEntry}.
			\mySec{Vorgehensweise}
			...weiter
 		\end{column}
%		Zweite Spalte
		\begin{column}{.49\textwidth}
		\mySec{weiter in der zweiten Spalte}
		Mit mehr Text
		%FILL - TEMP
		\vspace{50\baselineskip}
		%
		% Literatur
		\par{\color{uzlmain}Literatur}\par
 		{\small
			\bibliographystyle{abbrv}
 			\bibliography{literatur}
 		}
		\end{column}
	\end{columns}
\end{frame}
\end{document}